\problemname{Another Brick in the Wall}

The construction worker previously known as Lars has many bricks of
height $1$ and different lengths, and he is now trying to build a wall
of width $w$ and height $h$. Since the construction worker previously
known as Lars knows that the subset sum problem is $\mathsf{NP}$-hard,
he does not try to optimize the placement but he just lays the bricks
in the order they are in his pile and hopes for the best. First he
places the bricks in the first layer, left to right; after the first
layer is complete he moves to the second layer and completes it, and
so on. He only lays bricks horizontally, without rotating them. If at
some point he cannot place a brick and has to leave a layer
incomplete, then he gets annoyed and leaves. It does not matter if he
has bricks left over after he finishes.

Yesterday the construction worker previously known as Lars got really
annoyed when he realized that he could not complete the wall only at
the last layer, so he tore it down and asked you for help. Can you
tell whether the construction worker previously known as Lars will
complete the wall with the new pile of bricks he has today?

\section*{Input}

The first line contains three integers $h$, $w$, $n$
($1 \leq h \leq 100$, $1 \leq w \leq 100$, $1 \leq n \leq 10\,000$), the
height of the wall, the width of the wall, and the number of bricks
respectively. The second line contains $n$ integers $x_i$
($1 \leq x_i \leq 10$), the length of each brick.

\section*{Output}

Output \texttt{YES} if the construction worker previously known as
Lars will complete the wall, and \texttt{NO} otherwise.

%%% Local Variables:
%%% mode: latex
%%% TeX-master: "../../challenge-2016"
%%% End:
