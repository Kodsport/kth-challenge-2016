\problemname{nnnnn}

Hsara and Simone like to communicate without anyone else knowing what they're saying.
This time, Simone invented a very sneaky cipher.
When she wants to tell Hsara a non-negative number $n$, she performs the following encryption procedure.

Let $d(n)$ denote the decimal expansion of $n$.
Consider the string $x := d(n)^n$, i.e., the decimal expansion of $n$ concatenated with itself $n$ times.
The encryption of $n$ is then the length of $x$.

As an example, assume Simone wants to encrypt the number $10$.
Then
\[
        x = 10101010101010101010.
\]
The length of $x$ is then $20$, which will be the encrypted value of $x$.

Hsara had no problem writing a decryption algorithm for this procedure.
But can you?

\section*{Input}
The first and only line contains an integer $L$ ($0 \leq L \leq 10^{{10}^6}$), the encrypted value of some non-negative integer $n$.

\section*{Output}
Output a single line containing the integer $n$.
