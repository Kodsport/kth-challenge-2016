\problemname{Ironman}

\illustration{.3}{9552524790_2d9abf1580_k.jpg}{Cropped from picture by \href{https://www.flickr.com/photos/jblmmwr/9552524790/}{JBLM MWR}}

An ironman triathlon is a race where participants swim for $3.86$ km,
ride a bicycle for $180.25$ km, and finally run a marathon, and it is
considered one of the toughest sport events. Viveka has been training
for an even more challenging competition: the $n$-athlon. In an
$n$-athlon race, participants have to go from the starting point to
the finishing point through several types of terrain: water, sand,
ice, asphalt, etc. To make the race more interesting, participants are
free to pick the route that they think suits best their
abilities. Last year Viveka achieved an epic victory by skating the
last $40$ km in $1$ hour over ice, while her arch-rival Veronica was stuck in
a tar pit $1$ m from the finishing point.

The terrain distribution for this year has been published and now it
is your task as the optimization expert in Viveka's team to help her
figure out the best route for the race. The competition takes place in
a flat area, which we model as the 2D plane, and each type of terrain
is shaped as a horizontal strip. Participants are not allowed to leave
the race area. You know the position of each strip and Viveka's speed
in that type of terrain.

\section*{Input}
The first line contains two pairs of decimal numbers $x_s$, $y_s$,
$x_f$, $y_f$, the $x$ and $y$ coordinates of the starting and
finishing point, respectively, in meters. The second line contains one
integer $n$ ($1 \leq n \leq 10\,000$), the number of layers.  The
third line contains $n-1$ decimal numbers, the $y$ coordinate of each
change between layers.  Layers are given in order, this is,
$y_s < y_1 < y_2 < \cdots < y_{n-1} < y_f$, so the shape of layer $i$
is $(-10\,000,10\,000)\times(y_{i-1},y_{i})$. The first and last
layers extend only until the $y$ coordinate of the starting and
finishing point, this is they have shape
$(-10\,000,10\,000)\times(y_s,y_1)$ and
$(-10\,000,10\,000)\times(y_{n-1},y_f)$ respectively.
The fourth line contains $n$ decimal numbers, Viveka's speed in each
layer, in meters per second.  All decimal numbers have absolute value
at most $10^4$ and at most $4$ decimals.

\section*{Output}
Output the minimum time required for Viveka to go from the starting to
the finishing point. Your answer should be within absolute or relative
error at most $10^{-6}$.

%%% Local Variables:
%%% mode: latex
%%% TeX-master: "../../challenge-2016"
%%% End:
